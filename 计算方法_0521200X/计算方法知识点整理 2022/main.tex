\documentclass[fontset=ubuntu]{ctexart}

\usepackage{fancyhdr}
\usepackage{color}
\usepackage{amsmath}
\usepackage{amssymb}
\usepackage{geometry}
\usepackage{enumitem}
\usepackage{multicol}
\usepackage{indentfirst}
\usepackage{authblk}
\usepackage{draftwatermark}
\usepackage[hidelinks]{hyperref}

\pagestyle{fancy}

\fancyhf{}
\fancyhead[L]{计算方法复习\ 第三版(无目录)}
\fancyhead[R]{\leftmark}
\fancyhead[C]{eslzzyl@163.com}
\fancyfoot[C]{\thepage}

\SetWatermarkText{Eslzzyl整理}            % 设置水印内容
\SetWatermarkLightness{0.9}             % 设置水印透明度 0-1
\SetWatermarkScale{0.8}                   % 设置水印大小 0-1

\setlist[itemize,2]{label=$\diamond$}
\geometry{a4paper,scale=0.8}

\newtheorem{theorem}{定理}

\numberwithin{equation}{section}
\numberwithin{theorem}{section}

\ctexset{
    section={   
        name={,},
        number={\Roman{section}}
    }
}

\setlength\columnsep{0.6cm}
\columnseprule=0.5pt

\title{计算方法复习\ 第三版}
\author{Eslzzyl}
\affil{eslzzyl@163.com}

\begin{document}
    \maketitle
    \newpage
    \tableofcontents
    \vspace{20pt}
    \textcolor{red}{\textbf{请勿以任何形式出售本文档,或将本文档作为出售资料的赠品。因笔者水平有限,错漏在所难免,因参考本文档的错漏部分而导致考试丢分的,笔者概不负责。}}
    \newpage

\section{引论}
\begin{multicols}{2}

    \subsection{数值计算注意事项}
    \begin{itemize}
        \item 选择数值稳定的计算公式
        \item 避免两个相近的数相减
        \item 绝对值太小(接近零)的数不能作除数
        \item 避免大数吃掉小数
        \begin{itemize}
            \item 求和时,从小数加到大数,而不是反过来。
        \end{itemize}
        \item 简化计算步骤,减少运算次数,避免误差积累
        \item 控制舍入误差的累积和传播
    \end{itemize}

    有效的算法:运算量少,应用范围广,需用存储单元少,逻辑结构简单,便于编写计算机程序,而且计算结果可靠。
    
    \subsection{二分法}

    想要二分法达到误差不超过$10^{-m}$,则应该二分$k$次,且有
    \begin{equation*}
        2^{-k}<10^{-m},\quad k\text{取可能的最大值}
    \end{equation*}
    
    \subsection{误差}
    \subsubsection{来源(P9)}
    \begin{itemize}
        \item 计算误差
        \begin{itemize}
            \item 截断误差:求解方法本身的限制,如果是近似的数值解法,则不可能完全精确。
            \item 舍入误差:计算机字长有限,数据存储时不可能无穷精确。
        \end{itemize}
        \item 固有误差
        \begin{itemize}
            \item 模型误差(本课程不考虑)
            \item 测量误差(同上)
        \end{itemize}
    \end{itemize}
    \subsubsection{绝对误差限(P10)}

    绝对误差限/误差限/精度 $\epsilon$:满足$\lvert x-x^*\rvert\leq\epsilon$\par
    
    四舍五入得到的结果,其误差限为最末一位的半个单位。

    \subsubsection{有效数字(P10)}

    对$x^*$的规格化表示
    \begin{equation}
        x=\pm 0.a_1a_2\cdots a_n\times 10^m
    \end{equation}

    若
    \begin{equation}
        \lvert x-x^*\rvert\leq \frac{1}{2}\times10^{m-l},1\leq l\leq n
    \end{equation}
    则称近似值$x$有$l$位有效数字。

    准确值可认为有无穷位有效数字。

    有效数字相同,误差限不一定相同。如12345和0.12345都有5位有效数字,但前者误差限为0.5,后者为0.000005。
    
    \subsubsection{相对误差限(P11)}

    若
    \begin{equation}
        \frac{\lvert x-x^*\rvert}{x}\leq \epsilon
    \end{equation}
    则$\epsilon$是$x$的相对误差限。

\end{multicols}

%\newpage

\section{第一章\ 插值方法}
\begin{multicols}{2}

    内插:插值点在插值区间内的插值。

    外推:插值点在插值区间外的插值。不可靠。

    \subsection{拉格朗日插值}

    \subsubsection{线性插值(P16)}

    点斜式:
    \begin{equation}
        p_1(x) = y_0+\frac{y_1-y_0}{x_1-x_0}(x-x_0)
    \end{equation}

    \subsubsection{一般情形(P18)}

    插值基函数
    \begin{equation}
        l_k(x)=\prod_{j=0\atop j\neq k}^{n}\frac{x-x_j}{x_k-x_j}
    \end{equation}

    拉格朗日插值公式
    \begin{equation}
        p_n(x)=\sum_{k=0}^{n}l_k(x)y_k=\sum_{k=0}^{n}\left(\prod_{j=0\atop j\neq k}^{n}\frac{x-x_j}{x_k-x_j} \right )y_k
    \end{equation}
    $p_n(x)$也可表为$L_n(x)$,注意$L_n(x)$和$l_k(x)$不是同一回事。

    拉格朗日插值余项(即截断误差)
    \begin{equation}
        f(x)-p_n(x)=\frac{f^{(n+1)}(\xi)}{(n+1)!}\prod_{k=0}^{n}(x-x_k)
    \end{equation}
    其中$\xi\in[a,b]$

    \subsubsection{误差的事后估计(P21)}

    直接用计算结果估计误差的方法称为\textbf{事后估计法}。

    公式:
    \begin{equation}
        y-y_1\approx \frac{x-x_1}{x_2-x_1}(y_2-y_1)
    \end{equation}

    通过公式计算出估计误差后,可以将结果加上误差得到更精确的修正值。

    \subsection*{埃特金插值}

    本节为选讲,故略。

    \subsection{牛顿插值(P23)}

    \subsubsection{差商(P24)}

    一阶差商
    \begin{equation}
        f(x_0,x_1)=\frac{f(x_1)-f(x_0)}{x_1-x_0}
    \end{equation}

    二阶差商:即一阶差商的差商
    \begin{equation}
        f(x_0,x_1,x_2)=\frac{f(x_1,x_2)-f(x_0,x_1)}{x_2-x_0}
    \end{equation}

    推广到n阶差商:
    \begin{equation}
        \begin{split}
        &f(x_0,x_1,\cdots,x_n)=\\
        &\frac{f(x_1,x_2,\cdots,x_n)-f(x_0,x_1,\cdots,x_{n-1})}{x_n-x_0}
        \end{split}
    \end{equation}

    差商的值与节点的排列顺序无关,即差商的\textbf{对称性}。

    \begin{theorem}
        在$x_0,x_1,\cdots,x_n$所界定的范围$\Delta:\left[\underset{0\leq i\leq n}{\text{min}}x_i, \underset{0\leq i\leq n}{\text{max}}x_i \right]$内存在一点$\xi$,使成立
        \begin{equation*}
            f(x_0,x_1,\cdots,x_n)=\frac{f^{(n)}(\xi)}{n!}
        \end{equation*}
    \end{theorem}

    \subsubsection{差商形式的牛顿插值公式(P26)}

    公式
    \begin{equation}
        \begin{split}
        p_n(x)=\ &f(x_0)+f(x_0,x_1)(x-x_0)+\\
        &f(x_0,x_1,x_2)(x-x_0)(x-x_1)+\cdots+\\
        &f(x_0,x_1,\cdots,x_n)(x-x_0)(x-x_1)\cdots (x-x_{n-1})
        \end{split}
    \end{equation}

    余项(截断误差)
    \begin{equation}
        R(x)=f(x_0,x_1,\cdots,x_n,x)(x-x_0)(x-x_1)\cdots(x-x_n)
    \end{equation}

    \subsubsection{差分(P26)}

    关于函数值$f(x_i)=y_i$的\textbf{一阶差分}定义为
    \begin{equation*}
        \Delta y_i = y_{i+1}-y_i
    \end{equation*}
    \textbf{二阶差分}定义为一阶差分的差分
    \begin{equation*}
        \Delta^2 y_i = \Delta y_{i+1}-\Delta y_i
    \end{equation*}
    \textbf{n阶差分}定义为
    \begin{equation*}
        \Delta^w y_i = \Delta^{n-1}y_{i+1}-\Delta^{n-1}y_i
    \end{equation*}

    节点等距的情况下,差商可用差分来表示。设步长$h=x_{i+1}-x_i$,有
    \begin{equation*}
        f(x_i,x_{i+1},\cdots,x_{i+k}) = \frac{1}{k!h^k}\Delta^k y_i
    \end{equation*}

    \subsubsection{差分形式的牛顿插值公式}

    令$x=x_0+th$,则有
    \begin{align*}
        p_n(x_0+th)=&y_0+t\Delta y_0+\frac{t(t-1)}{2}\Delta^2 y_0+\cdots+ \\
        &\frac{t(t-1)(t-2)\cdots(t-n+1)}{n!}\Delta^n y_0
    \end{align*}
    这一公式称为函数插值的\textbf{有限差公式}。

    \subsection{埃尔米特插值(P28)}

    略

    \subsection{分段插值(P30)}

    当拉格朗日插值的插值次数n增大时,插值函数$p_n(x)$会在插值区间的两端发生激烈的震荡,称为\textbf{龙格现象}。
    龙格现象说明,在大范围内使用高次插值,逼近的效果往往是不理想的。

    分段插值的优缺点:

    \begin{itemize}
        \item 优点:显式算法,方法简单,收敛性好,有局部性,不易受到其他区间的影响。
        \item 缺点:需要各个节点的导数值,要求高;光滑性差。
    \end{itemize}

    \subsubsection{分段三次插值(P32)}

    公式:
    \begin{align}
        S_3(x)=&\varphi_0\left(\frac{x-x_i}{h_i} \right ) y_i+\varphi_1\left(\frac{x-x_i}{h_i} \right )y_{i+1}\nonumber \\
        &+h_i\psi _0\left(\frac{x-x_i}{h_i} \right )y_i'+h_i\psi_1\left(\frac{x-x_i}{h_i} \right )y_{i+1}'
    \end{align}
    其中$h_i=x_{i+1}-x_i$,而$\varphi_0,\varphi_1,\psi_0,\psi_1$由式(\ref{phi_and_psi})给出(看右边)。

    误差
    \begin{equation}
        \lvert f(x)-S_3(x)\rvert \leq \frac{h^4}{384}\underset{a\leq x\leq b}{\text{max}}\lvert f^{(4)}(x)\rvert
    \end{equation}

    \subsection{样条函数(P33)}

    称具有分划
    \begin{equation*}
        \varDelta:a=x_0<x_1<\cdots<x_n=b
    \end{equation*}
    的分段$k$次式$S_k(x)$为\textbf{$k$次样条函数}:如果它在每个内节点$x_i(1\leq i\leq n-1)$上具有直到$k-1$阶连续导数。
    点$x_i$称作样条函数的\textbf{节点}。

    样条函数简称样条,其特点是,既是充分光滑的,又保留有一定的间断性。

    样条插值其实是一种改进的分段插值。一次样条插值和一次分段插值是同一回事。(P33底部)

    \subsubsection{三次样条插值}

    原理比较复杂,只给出求解过程:

    首先计算$\alpha_i$和$\beta_i$
    \begin{gather}
        \alpha_i = \frac{h_{i-1}}{h_{i-1}+h_i} \\
        \beta_i = 3\left[(1-\alpha_i)\frac{y_i-y_{i-1}}{h_{i-1}}+\alpha_i\frac{y_{i+1}-y_i}{h_i}\right]
    \end{gather}
    其中$h_i=x_{i+1}-x_i$(所以$\alpha_i$的$i$是从1开始取的)。

    然后列出如下关于$m_i$的方程组
    \begin{equation}
        \left\{\begin{array}{l}
            2m_1+\alpha_1 m_2 = \beta_1-(1-\alpha_1)y_0' \\
            (1-\alpha_2)m_1+2m_2+\alpha_2 m_3 = \beta_2 \\
            \cdots\cdots \\
            (1-\alpha_{n-1})m_{n-3}+2m_{n-2}+\alpha_{n-2}m_{n-1} = \beta_{n-2} \\
            (1-\alpha_{n-1})m_{n-2}+2m_{n-1}=\beta_{n-1}-\alpha_{n-1}y_n'
        \end{array}\right.
    \end{equation}
    用追赶法(\ref{zhuiganfa})解方程组,得到$m_1,m_2,\cdots,m_{n-1}$

    然后判断$x$所在区间$[x_i,x_{i+1}]$,最后用下式
    \begin{align}
        S_3(x)=&\varphi_0\left(\frac{x-x_i}{h_i} \right ) y_i+\varphi_1\left(\frac{x-x_i}{h_i} \right )y_{i+1}\nonumber \\
        &+h_i\psi _0\left(\frac{x-x_i}{h_i} \right )m_i+h_i\psi_1\left(\frac{x-x_i}{h_i} \right )m_{i+1}
    \end{align}
    插出$y=S_3(x)$,得到结果。其中
    \begin{equation}
        \label{phi_and_psi}
        \left\{\begin{array}{l}
            \varphi_0(x) = (x-1)^2(2x+1) \\
        \varphi_1(x) = x^2(-2x+3) \\
        \psi_0(x) = x(x-1)^2 \\
        \psi_1(x) = x^2(x-1)
        \end{array}\right.
    \end{equation}

    \subsection{曲线拟合的最小二乘法(P36)}

    预测值$\widehat{y_i}$和实测值$y_i$的差称为\textbf{残差}:
    \begin{equation}
        e_i=y_i-\widehat{y_i}
    \end{equation}

    构造拟合曲线可选用如下三种准则之一:

    (1).使残差的最大绝对值最小:$\underset{i}{\text{max}}\lvert e_i\rvert=min$

    (2).使残差的绝对值之和最小:$\sum_{i}\lvert e_i\rvert=min$

    (3).使残差的平方和最小:$\sum_{i}e_i^2=min$

    其中使用(3)的方法称为最小二乘法。

    \subsubsection{直线拟合}

    \begin{equation}
        \left\{\begin{array}{l}
        aN+b\sum x_i=\sum y_i \\
        a\sum x_i +b\sum x_i^2=\sum x_i y_i
        \end{array}\right.
    \end{equation}

    \subsubsection{多项式拟合}

    拟合对象是$m$次多项式
    \begin{equation}
        y=\sum_{j=0}^{m}a_j x^j
    \end{equation}

    待定系数$a_0,a_1,\cdots,a_m$

    拟合的正则方程组如下:
    \begin{equation}
        \left\{\begin{array}{l}
            a_0N+a_1\sum x_i+\cdots+a_m\sum x_i^m=\sum y_i \\
            a_0\sum x_i+a_1\sum x_i^2+\cdots+a_m\sum x_i^{m+1}=\sum x_i y_i \\
            \cdots \\
            a_0\sum x_i^m +a_1\sum x_i^{m+1}+\cdots+a_m\sum x_i^{2m}=\sum x_i^m y_i
        \end{array}\right.
    \end{equation}

    % \subsubsection*{观察数据的修匀}

    % 选讲,故略。

\end{multicols}

% \newpage

\section{第二章\ 数值积分}

\begin{multicols}{2}
    \subsection{机械求积}

    \subsubsection{数值求积的基本思想}

    根据积分中值定理,存在一点$\xi\in[a,b]$,使得
    \begin{equation}
        \int_{a}^{b}f(x)dx=(b-a)f(\xi)
    \end{equation}
    其中$f(\xi)$称为区间$[a,b]$上的平均高度,根据这个平均高度的估计方法的不同,可以得到一些公式。

    梯形公式
    \begin{equation}
        \int_{a}^{b}f(x)dx\approx \frac{b-a}{2}\left[f(a)+f(b) \right ]
    \end{equation}

    中矩形公式
    \begin{equation}
        \int_{a}^{b}f(x)dx\approx (b-a)f\left(\frac{a+b}{2} \right )
    \end{equation}

    辛甫生公式
    \begin{equation}
        \int_{a}^{b}f(x)dx\approx \frac{b-a}{6}\left[f(a)+4f\left(\frac{a+b}{2} \right )+f(b) \right ]
    \end{equation}

    更一般地,取$[a,b]$内若干个点$x_k$处的高度$f(x_k)$,加权平均得到平均高度$f(\xi)$,这类公式的一般形式是
    \begin{equation}
        \label{qiuji_tongshi}
        \int_{a}^{b}f(x)dx\approx \sum_{k=0}^{n}A_kf(x_k)
    \end{equation}
    其中$x_k$称为\textbf{求积节点},$A_k$称为\textbf{求积系数}。

    \subsubsection{代数精度(P59)}

    若公式(\ref{qiuji_tongshi})对于一切次数$\leq m$的多项式是准确的,但对于$m+1$次多项式(指$x$的次数)不准确,就称它具有\textbf{$m$次代数精度}。

    梯形公式和中矩形公式有一次代数精度,辛甫生公式有三次代数精度。

    求给定插值公式的代数精度:令其对$f(x)=1,f(x)=x,f(x)=x^2,\cdots,f(x)=x^n$准确成立,找到使得对$f(x)=x^i$准确成立而对$f(x)=x^{i+1}$不准确成立的$i$值,即为所求值。

    \subsubsection{插值型的求积公式(P60)}

    对于式(\ref{qiuji_tongshi}),如果其所有$A_k$均满足
    \begin{equation}
        A_k = \int_{a}^{b}l_k(x)dx
    \end{equation}
    其中$l_k(x)$有如下形式
    \begin{equation*}
        l_k(x) = \prod_{j=0\atop j\neq k}^{n}\frac{x-x_j}{x_k-x_j}
    \end{equation*}
    则这样的求积公式是\textbf{插值型}的。

    \begin{theorem}
        \label{chazhi}
        形如(\ref{qiuji_tongshi})的求积公式至少具有$n$次代数精度的充要条件是:它是插值型的。
    \end{theorem}

    \subsection{牛顿-柯特斯公式(P61)}

    设分$[a,b]$为$n$等分,步长$h=\frac{b-a}{n}$,取等分点$x_k=a+kh\ (k=0,1,\cdots,n)$构造出的插值型求积公式
    \begin{equation}
        I_n = (b-a)\sum_{k=0}^{n}C_kf(x_k)
    \end{equation}
    称作\textbf{牛顿-柯特斯公式},其中$C_k$称为\textbf{柯特斯系数}。

    一阶牛顿-柯特斯公式就是梯形公式。

    二阶牛顿-柯特斯公式就是辛甫生公式。

    四阶牛顿-柯特斯公式
    \begin{align}
        C=&\frac{b-a}{90}\Bigl[ 7f(x_0)+32f(x_1) \nonumber \\
        &+12f(x_2)+32f(x_3)+7f(x_4)\Bigr]
    \end{align}
    称为\textbf{柯特斯公式}。

    更高阶的牛顿-柯特斯公式不稳定,一般不用。一般令$n\leq 7$。

    由定理\ref{chazhi}知,$n$阶牛顿-柯特斯公式\textbf{至少}具有$n$次代数精度。实际上辛甫生公式(二阶)和柯特斯公式(四阶)分别具有3次和5次代数精度。

    \subsubsection{复化求积(P63)}

    复化求积法是指,先用低阶的求积公式求得每个子段$[x_k,x_{k+1}]$上的积分值$I_k$,然后将它们累加求和,用$\sum_{k=0}^{n-1}I_k$作为积分$I$的近似值。

    在下面的一系列式子中,都有$h=x_{k+1}-x_k$。

    复化梯形公式
    \begin{equation}
        T_n=\frac{h}{2}\left[f(a)+2\sum_{k=1}^{n-1}f(x_k)+f(b)\right]
    \end{equation}
    误差
    \begin{equation}
        I-T_n\approx -\frac{h^2}{12}\left[f'(b)-f'(a)\right]
    \end{equation}

    复化辛甫生公式
    \begin{equation}
        S_n=\frac{h}{6}\left[f(a)+4\sum_{k=0}^{n-1}f(x_{k+\frac{1}{2}})+2\sum_{k=1}^{n-1}f(x_k)+f(b)\right]
    \end{equation}
    误差
    \begin{equation}
        I-S_n\approx -\frac{1}{180}\left(\frac{h}{2}\right)^4\left[f'''(b)-f'''(a)\right]
    \end{equation}

    复化柯特斯公式
    \begin{align}
        C_n=& \frac{h}{90} \biggl[ 7f(a)+32\sum_{k=0}^{n-1} f(x_{k+\frac{1}{4}}) \nonumber \\
        &+12 \sum_{k=0}^{n-1} f(x_{k+\frac{1}{2}}) + 32\sum_{k=0}^{n-1} f(x_{k+\frac{3}{4}}) \nonumber \\
        &+14 \sum_{k=1}^{n-1} f(x_k)+7f(b) \biggr]
    \end{align}
    误差
    \begin{equation}
        I-C_n\approx -\frac{2}{945}\left(\frac{h}{4}\right)^6\left[f^{(5)}(b)-f^{(5)}(a)\right]
    \end{equation}

    \subsection{龙贝格算法(P66)}

    根据事后误差估计法修正上节的公式(过程见P68),可以得到如下迭代加速公式:
    \begin{gather}
        S_n = \frac{4}{3}T_{2n}-\frac{1}{3}T_n \\
        C_n = \frac{16}{15}S_{2n}-\frac{1}{15}S_n \\
        R_n = \frac{64}{63}C_{2n}-\frac{1}{63}C_n
    \end{gather}

    于是,根据$T_1,T_2,T_4,T_8,T_{16},\cdots$,可以得到$S_1,S_2,S_4,S_8,\cdots$,可以得到$C_1,C_2,C_4,\cdots$,可以得到$R_1,R_2,\cdots$。
    这种加速方法称为\textbf{龙贝格算法}。注意它已经不属于牛顿-柯特斯公式。

    \subsection{高斯公式(P71)}

    牛顿-柯特斯公式的求积节点是等分的,如果能够适当地选取这些节点,可以使求积公式具有$2n-1$次代数精度。这种公式称为\textbf{高斯公式},相应的求积节点$x_k$称为\textbf{高斯点}。通式如下:
    \begin{equation}
        \int_{-1}^{1}f(x)dx\approx \sum_{k=1}^{n}A_k f(x_k)
    \end{equation}

    高斯公式也是插值型求积公式。(课本P71指出“本章所考察的求积公式都是插值型的”)

    一点高斯公式就是中矩形公式($2\times 1-1=1$次代数精度):
    \begin{equation}
        \int_{-1}^{1}f(x)dx\approx 2f(0)
    \end{equation}
    其高斯点$x_1=0$。

    两点高斯公式($2\times 2-1=3$次代数精度):
    \begin{equation}
        \label{liangdian_gaosi}
        \int_{-1}^{1}f(x)dx\approx f\left(-\frac{1}{\sqrt{3}}\right)+f\left(\frac{1}{\sqrt{3}}\right)
    \end{equation}
    其高斯点$x_1=-\frac{1}{\sqrt{3}},x_2=\frac{1}{\sqrt{3}}$。

    对上式(\ref{liangdian_gaosi})做变换,可以得到$[a,b]$上的高斯公式:
    \begin{align}
        \int_{-1}^{1}f(x)dx\approx &\frac{b-a}{2}\biggl[ f\left(-\frac{b-a}{2\sqrt{3}}+\frac{a+b}{2}\right) \nonumber \\
        &+f\left(\frac{b-a}{2\sqrt{3}}+\frac{a+b}{2}\right)\biggr]
    \end{align}

    三点高斯公式($2\times 3-1=5$次代数精度):
    \begin{equation}
        \int_{-1}^{1}f(x)dx\approx \frac{5}{9}f\left(-\sqrt{\frac{3}{5}}\right)+\frac{8}{9}f(0)+\frac{5}{9}f\left(\sqrt{\frac{3}{5}}\right)
    \end{equation}

    \subsubsection{高斯点的基本特性(P72)}

    \begin{theorem}
        节点$x_k\ (k=1,2,\cdots,n)$是高斯点的充要条件是,多项式
        \begin{equation*}
            \omega(x)=(x-x_1)(x-x_2)\cdots(x-x_n)
        \end{equation*}
        满足
        \begin{equation*}
            \int_{-1}^{1}x^k\omega(x)dx = 0,\quad k=0,1,\cdots,n-1
        \end{equation*}
    \end{theorem}

\end{multicols}

% \newpage

\section{第三章\ 常微分方程的差分方法}

\begin{multicols}{2}
    \subsection{欧拉方法}

    \subsubsection{欧拉格式(P98)}

    微分方程的数值解法首先就是要消除导数项,这一步称为\textbf{离散化}。

    微分方程可以写成
    \begin{equation}
        y'=f(x,y)
    \end{equation}
    的形式。用差商代替导数,整理可得\textbf{欧拉(Euler)格式}:
    \begin{equation}
        \label{xianshi_euler}
        y_{n+1}=y_n+hf(x_n,y_n),\quad n=0,1,2,\cdots
    \end{equation}

    在$y_n$为准确(即$y_n=y(x_n)$)的前提下估计误差$y(x_{n+1})-y_{n+1}$。这种误差称为\textbf{局部截断误差}。

    称一种数值方法的精度是$p$阶的,如果其局部截断误差为$O(h^{p+1})$。

    欧拉格式为一阶方法。

    \subsubsection{隐式欧拉格式(P99)}

    若将微分方程$y'(x_{n+1})=f(x_{n+1},y(x_{n+1}))$中的导数项用向后差商替代,则可得到
    \begin{equation}
        \label{yinshi_euler}
        y_{n+1}=y_n+hf(x_{n+1},y_{n+1})
    \end{equation}
    称为隐式欧拉格式。也是一阶方法。

    \subsubsection{两步欧拉格式(P99)}

    用中心差商替代导数项,可以得到两步欧拉格式
    \begin{equation}
        y_{n+1}=y_{n-1}+2hf(x_n,y_n)
    \end{equation}
    这是一种二阶方法。

    \subsection{改进的欧拉方法}

    思路是通过两端积分,将微分方程的导数项转换为积分项,然后采用数值求积方法解决。

    \subsubsection{梯形格式(P100)}

    公式如下:
    \begin{equation}
        \label{tixinggeshi}
        y_{n+1}=y_n+\frac{h}{2}\left[f(x_n,y_n)+f(x_{n+1},y_{n+1})\right]
    \end{equation}

    梯形格式是显式欧拉格式(\ref{xianshi_euler})和隐式欧拉格式(\ref{yinshi_euler})的算术平均。这是一种隐式格式。

    梯形格式具有二阶精度(存疑)。

    \subsubsection{改进的欧拉格式}

    先用欧拉方法求一个$y_{n+1}$的近似值,然后代替式\ref{tixinggeshi}右端的$y_{n+1}$,得到校正值,合起来写是
    \begin{equation}
        y_{n+1}=y_n+\frac{h}{2}\left[f(x_n,y_n)+f(x_{n+1},y_n+hf(x_n,y_n))\right]
    \end{equation}
    称为改进的欧拉格式。这是一种一步显式格式。

    \subsection{龙格-库塔方法}

    \subsubsection{设计思想}

    如果设法在$[x_n,x_{n+1}]$内多预报几个点的斜率值,然后将它们加权平均作为区间上的平均斜率$K^*$,则可能得到更高精度的方法。

    \subsubsection{二阶龙格-库塔方法(P103)}

    二阶龙格-库塔方法有两种常见形式(解释见课本)。第一种形式就是改进的欧拉格式。第二种格式又称\textbf{中点格式}:
    \begin{equation}
        \left\{\begin{array}{l}
            y_{n+1} = y_n+hK_2 \\
            K_1 = f(x_n,y_n) \\
            K_2 = f\left(x_{n+\frac{1}{2}},y_n+\frac{h}{2}K_1 \right )
        \end{array}\right.
    \end{equation}

    \subsubsection{三阶龙格-库塔方法(P104)}

    其中的一种:
    \begin{equation}
        \left\{\begin{array}{l}
            y_{n+1} = y_n+\frac{h}{6}(K_1+4K_2+K_3) \\
            K_1 = f(x_n,y_n) \\
            K_2 = f\left(x_{n+\frac{1}{2}},y_n+\frac{h}{2}K_1 \right ) \\
            K_3 = f\left(x_{n+1},y_n+h(-K_1+2K_2) \right )
        \end{array}\right.
    \end{equation}

    \subsubsection{四阶龙格-库塔方法(P105)}

    其中的一种:
    \begin{equation}
        \left\{\begin{array}{l}
            y_{n+1} = y_n=\frac{h}{6}(K_1+2K_2+2K_3+K_4) \\
            K_1 = f(x_n,y_n) \\
            K_2 = f\left(x_{n+\frac{1}{2}},y_n+\frac{h}{2}K_1 \right ) \\
            K_3 = f\left(x_{n+\frac{1}{2}},y_n+\frac{h}{2}K_2 \right ) \\
            K_4=f\left(x_{n+1},y_n+hK_3\right)
        \end{array}\right.
    \end{equation}
    
    龙格-库塔方法要求解具有良好的光滑性。如果光滑性差,那么龙格-库塔方法可能不如改进的欧拉方法。

    \subsection{亚当姆斯方法(P108)}

    (本节疑似不考)

    记$y_{n-k}'=f(x_{n-k},y_{n-k})$(差商),则有如下显式格式:\\
    二阶显式亚当姆斯格式
    \begin{equation}
        y_{n+1}=y_n+\frac{h}{2}(3y_n'-y_{n-1}')
    \end{equation}
    三阶显式亚当姆斯格式
    \begin{equation}
        y_{n+1}=y_n+\frac{h}{12}(23y_n'-16y_{n-1}'+5y_{n-2}')
    \end{equation}
    四阶显式亚当姆斯格式
    \begin{equation}
        y_{n+1}=y_n+\frac{h}{24}(55y_n'-59y_{n-1}'+37y_{n-2}'-9y_{n-3}')
    \end{equation}

    此外还有如下隐式格式:\\
    二阶隐式亚当姆斯格式
    \begin{equation}
        y_{n+1}=y_n+\frac{h}{2}(y_{n+1}'+y_n')
    \end{equation}
    三阶隐式亚当姆斯格式
    \begin{equation}
        y_{n+1}=y_n+\frac{h}{12}(5y_{n+1}'+8y_n'-y_{n-1}')
    \end{equation}
    四阶隐式亚当姆斯格式
    \begin{equation}
        y_{n+1}=y_n+\frac{h}{24}(9y_{n+1}'+19y_n'-5y_{n-1}'+y_{n-2}')
    \end{equation}

    \subsubsection{亚当姆斯预报-校正系统(P109)}

    类似改进的欧拉方法,通过显式、隐式亚当姆斯格式可以得到\textbf{亚当姆斯预报-校正系统},下面是四阶的情况:\\
    预报
    \begin{align*}
        & \overline{y}_{n+1}=y_n+\frac{h}{24}(55y_n'-59y_{n-1}'+37y_{n-2}'-9y_{n-3}') \\
        & \overline{y}_{n+1}'=f(x_{n+1},y_{n+1})
    \end{align*}
    校正
    \begin{align*}
        & y_{n+1}=y_n+\frac{h}{24}(9\overline{y}_{n+1}'+19y_n'-5y_{n-1}'+y_{n-2}') \\
        & y_{n+1}'=f(x_{n+1},y_{n+1})
    \end{align*}

    使用时,$y_{n+1}$是输出的结果,$y_{n+1}'$用于为后续插值点提供数据。

    此预报-校正系统的事后误差估计与补偿公式请见课本P111。

    \subsection{收敛性与稳定性(P112)}

    主要内容略。

    \subsubsection{收敛性问题}

    对于任意固定的$x_n=x_0+nh$,如果数值解$y_n$当$h\to 0$(同时$n\to\infty$)时趋向于准确解$y(x_n)$,则称该方法是\textbf{收敛}的。

    \subsubsection{稳定性问题}

    欧拉方法是条件稳定的。

    隐式欧拉格式是恒稳定(无条件稳定)的。

\end{multicols}

% \newpage

\section{第四章\ 方程求根的迭代法}

\begin{multicols}{2}
    \subsection{迭代过程的收敛性}

    对一般方程$f(x)=0$,为了使用迭代法,需要将其改写成$x=\varphi(x)$的格式,其中$\varphi(x)$成为迭代函数。

    迭代公式:$x_{k+1}=\varphi(x_k),\quad k=0,1,2,\cdots$

    如果$x_k$有极限,则称\textbf{迭代收敛}。

    几何意义:求根问题在几何上就是确定曲线$y=\varphi(x)$与直线$y=x$的交点。

    \subsubsection{压缩映像原理(P129)}

    \begin{theorem}
        设$\varphi(x)$在$[a,b]$上具有连续的一阶导数,且满足下列两项条件:\\
        (1).对于任意$x\in[a,b]$,总有$\varphi\in[a,b]$;\\
        (2).存在$0\leq L\leq 1$,使对于任意$x\in[a,b]$成立\\
        \begin{equation}
            \lvert\varphi'(x)\rvert \leq L
        \end{equation}
        则迭代过程$x_{k+1}=\varphi(x_k)$对任意初值$x_0\in[a,b]$均收敛于方程$x=\varphi(x)$的根$x^*$,且有下列误差估计式\\
        \begin{gather}
            \lvert x^*-x_k\rvert\leq \frac{1}{1-L}\lvert x_{k+1}-x_k\rvert \\
            \lvert x^*-x_k\rvert\leq\frac{L^k}{1-L}\lvert x_1-x_0\rvert
        \end{gather}
    \end{theorem}

    \subsubsection{局部收敛性}

    称一种迭代过程在根$x^*$\textbf{邻近收敛},如果存在邻域$\varDelta$:$\lvert x-x^*\rvert\leq\delta$,使迭代过程
    对于任意初值$x_0\in\varDelta$均收敛。这种在根的邻近所具有的收敛性称为\textbf{局部收敛性}。

    \begin{theorem}
        设$\varphi(x)$在$x\varphi(x)$的根$x^*$邻近有连续一阶导数,且成立 \\
        \begin{equation*}
            \lvert\varphi'(x^*)\rvert \leq 1
        \end{equation*}
        则迭代过程$x_{k+1}=\varphi(x_k)$在$x^*$邻近具有局部收敛性。
    \end{theorem}

    \subsubsection{收敛速度}

    如果迭代误差$e_k=x^*-x_k$当$k\rightarrow\infty$时成立
    \begin{equation*}
        \frac{e_{k+1}}{e_k^p}\rightarrow c\quad (c\neq 0\ \text{常数})
    \end{equation*}
    则称迭代过程是\textbf{$p$阶收敛的}。$p=1$时称\textbf{线性收敛},$p=2$时称平方收敛。

    \begin{theorem}
        设$\varphi(x)$在$x=\varphi(x)$的根$x^*$邻近有连续二阶导数,且 \\
        \begin{equation*}
            \lvert\varphi'(x^*)\rvert \leq 1
        \end{equation*}
        则$\varphi'(x^*)\neq0$时迭代过程为线性收敛;而当$\varphi'(x^*)=0$,$\varphi''(x^*)\neq0$时为平方收敛。
    \end{theorem}

    \subsection{迭代过程的加速}

    \subsubsection{经典加速方法(P133)}

    加速迭代公式(P133)
    \begin{equation}
        x_{k+1}=\frac{1}{1-L}\left[\varphi(x_k)-Lx_k\right]
    \end{equation}
    其中$L=\varphi'(x_0)$。

    \subsubsection{埃特金加速方法(P133)}

    经典方法的$L$需要求导得到,此法的好处是不需要求导。

    迭代\ $\overline{x}_{k+1}=\varphi(x_k)$

    迭代\ $\widetilde{x}_{k+1}=\varphi\left(\overline{x}_{k+1} \right )$

    改进\ $x_{k=1}=\widetilde{x}_{k+1}-\frac{(\widetilde{x}_{k+1}-\overline{x}_{k+1})^2}{\widetilde{x}_{k+1}-2\overline{x}_{k+1}+x_k}$

    \subsection{牛顿法(P135)}

    牛顿公式:
    \begin{equation}
        x_{k+1}=x_k-\frac{f(x_k)}{f'(x_k)}
    \end{equation}

    几何解释:P136

    \begin{theorem}
        牛顿法在$f(x)=0$的单根$x^*$附近平方收敛。
    \end{theorem}

    \subsubsection{开方公式(P137)}

    使用牛顿法解二次方程$x^2-c=0$即得开方公式:
    \begin{equation}
        x_{k+1}=\frac{1}{2}\left(x_k+\frac{c}{x_k}\right)
    \end{equation}

    \begin{theorem}
        开方公式对于任意初值$x_0>0$均平方收敛。
    \end{theorem}

    \subsubsection{牛顿下山法(P138)}

    若初值$x_0$偏离$x^*$较远,则牛顿法可能发散。为了防止发散,需要额外保证函数值单调下降:
    \begin{equation}
        \lvert f(x_{k=1})\rvert<\lvert f(x_k)\rvert
    \end{equation}

    因此采用下列迭代公式
    \begin{equation}
        x_{k+1}=x_k-\lambda\frac{f(x_k)}{f'(x_k)}
    \end{equation}
    其中$0<\lambda\leq1$称为\textbf{下山因子}。

    操作方法:从$\lambda=1$开始反复将$\lambda$的值减半试算,一旦上面的单调性条件成立,则认为“下山成功”。反之若找不到合适的下山因子,则认为“下山失败”,此时需重选$x_0$再下山。

    \subsection{弦截法(P139)}

    弦截法的思想是,用差商$\frac{f(x_k)-f(x_0)}{x_k-x_0}$代替牛顿法中的导数项。

    公式:
    \begin{equation}
        x_{k+1}=x_k-\frac{f(x_k)}{f(x_k)-f(x_0)}(x_k-x_0)
    \end{equation}
    
    弦截法为线性收敛。

    如果使用差商$\frac{f(x_k)-f(x_{k-1})}{x_k-x_{k-1}}$替代牛顿公式中的导数项,则得到\textbf{快速弦截法}:
    \begin{equation}
        x_{k+1}=x_k-\frac{f(x_k)}{f(x_k)-f(x_{k-1})}(x_k-x_{k-1})
    \end{equation}
    这是一种两步方法,使用时需要先提供两个开始值$x_0,x_1$。

\end{multicols}

% \newpage

\section{第五章\ 线性方程组的迭代法}

\begin{multicols}{2}
    \subsection{雅可比迭代、高斯-塞德尔迭代、超松弛法}

    这部分参考课本P156-161,结合例子容易理解。

    松弛法:
    \begin{equation}
        x_i^{(k+1)}=(1-\omega)x_i^{(k)}+\omega\times \text{G-S迭代值}
    \end{equation}
    其中$1<\omega<2$的松弛法称为\textbf{超松弛法(SOR方法)}。

    \subsection{向量和矩阵的范数}

    \subsubsection{向量的范数(P162)}

    任给向量$\mathbf{x}=(x_1,x_2,\cdots,x_n)^T$,其\textbf{范数}记$\lVert x\rVert$,它是一个实数,且满足下列三项条件:

    \begin{enumerate}[label=(\arabic*)]
        \item 对于任意向量$\mathbf{x}$,$\lVert \mathbf{x}\rVert\geq 0$,当且仅当$\mathbf{x}=\mathbf{0}$时$\lVert \mathbf{x}\rVert =0$。
        \item 对于任意实数$\lambda$及任意向量$\mathbf{x}$
            \begin{equation*}
                \lVert\lambda \mathbf{x}\rVert=\lvert\lambda\rvert\cdot \lVert \mathbf{x}\rVert
            \end{equation*}
        \item\label{sanjiao_budengshi} 对于任意向量$\mathbf{x}$和$\mathbf{y}$,有
            \begin{equation*}
                \lVert \mathbf{x}+\mathbf{y}\rVert\leq \lVert \mathbf{x}\rVert+\lVert \mathbf{y}\rVert
            \end{equation*}
    \end{enumerate}
    其中,性质\ref{sanjiao_budengshi}称作向量范数的\textbf{三角不等式}。

    $p$\ -\ 范数通式:
    \begin{equation}
        \lVert\mathbf{x}\rVert_p=\left(\sum_{i=1}^{n}\lvert x_i\rvert^p\right)^{\frac{1}{p}}
    \end{equation}

    常用范数:
    \begin{enumerate}[label=(\arabic*)]
        \item 2\ -\ 范数(长度)
            \begin{equation*}
                \lVert\mathbf{x}\rVert_2=\left(\sum_{i=1}^{n} x_i^2\right)^{\frac{1}{2}}
            \end{equation*}
        \item 1\ -\ 范数
            \begin{equation*}
                \lVert\mathbf{x}\rVert_1=\sum_{i=1}^{n}\lvert x_i\rvert
            \end{equation*}
        \item $\infty$\ -\ 范数
            \begin{equation*}
                \lVert\mathbf{x}\rVert_\infty=\underset{1\leq i\leq n}{\text{max}}\lvert x_i\rvert
            \end{equation*}
    \end{enumerate}

    \begin{theorem}
        对于任意向量$\mathbf{x}$,有
        \begin{equation}
            \lim_{p\to\infty}\lVert\mathbf{x}\rVert_p=\lVert\mathbf{x}\rVert_\infty
        \end{equation}
    \end{theorem}

    如果存在正数$c_1,c_2$,使对于任意向量$\mathbf{x}$均有
    \begin{equation*}
        \lVert\mathbf{x}\rVert_p\leq c_1\lVert\mathbf{x}\rVert_p,\quad \lVert\mathbf{x}\rVert_q\leq c_2\lVert\mathbf{x}\rVert_p
    \end{equation*}
    则称范数$\lVert\mathbf{\cdot}\rVert_p$与$\lVert\mathbf{\cdot}\rVert_q$等价。范数的等价关系有\textbf{传递性}。
    
    \begin{theorem}
        任何范数$\lVert\mathbf{x}\rVert_p\ (p<\infty)$均与$\lVert\mathbf{x}\rVert_\infty$等价,因而任何两种$p$\ -\ 范数彼此都是等价的。
    \end{theorem}

    \begin{theorem}
        在空间$\mathbb{R}^n$中,向量序列$\{\mathbf{x}^{(k)}\}$收敛于向量$\mathbf{x}$的充要条件是对$\mathbf{x}$的任意范数$\lVert\mathbf{\cdot}\rVert$,有:
        \begin{equation*}
            \lim_{k\to\infty}\lVert\mathbf{x}^{(k)}-\mathbf{x}^*\rVert_p=0
        \end{equation*}
    \end{theorem}

    \subsubsection{矩阵的范数(P164)}

    设$\mathbf{A}$是$n$阶方阵,$\lVert\mathbf{x}\rVert$为某范数,称
    \begin{equation*}
        \lVert\mathbf{A}\rVert=\underset{x\neq 0}{\text{max}}\frac{\lVert\mathbf{Ax}\rVert}{\lVert\mathbf{x}\rVert}
    \end{equation*}
    为矩阵$\mathbf{A}$的从属于该向量范数的范数,或称矩阵$\mathbf{A}$的范数,记为$\lVert\mathbf{A}\rVert$。

    矩阵范数的性质:
    \begin{enumerate}[label=(\arabic*)]
        \item (正定性)对任意方阵$\mathbf{A}$,$\lVert\mathbf{A}\rVert\geq 0$,当且仅当$\mathbf{A}=\mathbf{0}$时$\lVert\mathbf{A}\rVert=0$。
        \item (齐次性)对任意实数$\lambda$和任意方阵$\mathbf{A}$,有
            \begin{equation*}
                \lVert\lambda\mathbf{A}\rVert=\lvert\lambda\rvert\lVert\mathbf{A}\rVert
            \end{equation*}
        \item 对任意两个同阶方阵$\mathbf{A}$和$\mathbf{B}$,有
            \begin{gather*}
                \lVert\mathbf{A}+\mathbf{B}\rVert \leq \lVert\mathbf{A}\rVert + \lVert\mathbf{B}\rVert\quad \text{(三角不等式)}\\
                \lVert\mathbf{AB}\rVert \leq \lVert\mathbf{A}\rVert \lVert\mathbf{B}\rVert\quad \text{(相容性)}
            \end{gather*}
    \end{enumerate}

    有什么样的向量范数,就有什么样的矩阵范数。因此有矩阵$\mathbf{A}$的$p$\ -\ 范数:(略)

    矩阵$\mathbf{A}$的\textbf{行范数}:
    \begin{equation}
        \lVert\mathbf{A}\rVert_\infty=\underset{1\leq i\leq n}{\text{max}}\sum_{j=1}^{n}\lvert a_{ij}\rvert
    \end{equation}

    矩阵$\mathbf{A}$的\textbf{列范数}:
    \begin{equation}
        \lVert\mathbf{A}\rVert_1=\underset{1\leq j\leq n}{\text{max}}\sum_{i=1}^{n}\lvert a_{ij}\rvert
    \end{equation}

    矩阵$\mathbf{A}$的\textbf{F-范数}:
    \begin{equation}
        \lVert\mathbf{A}\rVert_F=\left(\sum_{j=1}^{n}\sum_{i=1}^{n}a_{ij}^2\right)^{\frac{1}{2}}
    \end{equation}

    矩阵$\mathbf{A}$的\textbf{2-范数(谱范数)}:
    \begin{equation}
        \lVert\mathbf{A}\rVert_2=\left[\lambda_\text{max}(A^T A)\right]^\frac{1}{2}
    \end{equation}
    其中,$\lambda_\text{max}$是矩阵$\mathbf{A}$的最大特征值。矩阵$\mathbf{A}$为对称矩阵时,它的2-范数和谱半径相等。

    易错:不要忽略了绝对值符号!

    \subsection{迭代过程的收敛性}

    雅可比迭代和高斯-塞德尔迭代公式都可以写成如下形式
    \begin{equation}
        \label{diedai_in_matrix}
        \mathbf{x}^{(k+1)}=\mathbf{GX}^{(k)}+\mathbf{b}
    \end{equation}
    其中$\mathbf{G}$称为公式()\ref{diedai_in_matrix})的\textbf{迭代矩阵}。

    \begin{theorem}
        若迭代矩阵$\mathbf{G}$满足
        \begin{equation}
            \lVert\mathbf{G}\rVert<1
        \end{equation}
        则迭代公式()\ref{diedai_in_matrix})对于任意初值$\mathbf{x}^{(0)}$均收敛。
    \end{theorem}

    \subsubsection{对角占优方程组}

    如果$n$阶方阵$\mathbf{A}=(a_{ij})_{n\times n}$的主对角线元素的绝对值大于同行其他元素的绝对值之和,则称其为\textbf{对角占优矩阵}。
    系数矩阵为对角占优矩阵的线性方程组称为\textbf{对角占优方程组}。

    \begin{theorem}
        若$A$为对角占优矩阵,则它是非奇异的,且它对应的线性方程组的雅可比迭代公式和高斯-塞德尔迭代公式都收敛。
    \end{theorem}

\end{multicols}

% \newpage

\section{第六章\ 线性方程组的直接法}

\begin{multicols}{2}
    \subsection{消去法}

    \subsubsection{约当消去法(P172)}

    约当消去法:每一步在一个方程中保留某个变元,而从其他方程中消去这个变元,反复消元后,方程组的每个方程都只剩下一个变元。
    约当消去法的总计算量约为$\frac{n^3}{3}$,$n$为方程个数。

    \subsubsection{高斯消去法(P176)}

    这是约当消去法的改进版本。计算量约为约当消去的50\%左右。

    \subsubsection{选主元素(P179)}

    略
    
    \subsection{追赶法}\label{zhuiganfa}

    有如下\textbf{三对角阵}:
    \begin{equation}
        \label{sanduijiaozhen}
        \mathbf{A}=
        \begin{bmatrix}
            &b_1  &c_1 &\cdots &\cdots &0 \\ 
            &a_2  &b_2 &c_2 &\cdots &\vdots \\
            &\vdots &\ddots &\ddots &\ddots &\vdots \\
            &\vdots &\cdots &a_{n-1} &b_{n-1} &c_{n-1} \\
            &0 &\cdots &\cdots &a_n &b_n 
        \end{bmatrix}
    \end{equation}

    \begin{theorem}
        假设上面的三对角阵(\ref{sanduijiaozhen})为对角占优,则它是非奇异的,且以(\ref{sanduijiaozhen})为系数矩阵的方程组有唯一解。
    \end{theorem}

    \subsubsection{追赶法的计算公式(P182)}

    \textbf{追的过程}(消元过程):按照如下顺序计算系数$u_1\to u_2\to\cdots\to u_{n-1}$和$y_1\to y_2\to\cdots\to y_n$。
    \begin{align}
        & u_1 = \frac{c_1}{b_1},\quad y_1=\frac{f_1}{b_1} \nonumber \\
        & u_i = \frac{c_i}{b_i-u_{i-1}a_i},\quad i = 2,3,\cdots,n-1 \\
        & y_i = \frac{f_i-y_{i-1}a_i}{b_i-u_{i-1}a_i},\quad i = 2,3,\cdots,n \nonumber
    \end{align}
    $a_i,b_i,c_i$都是矩阵(\ref{sanduijiaozhen})中的值,$f_i$是对应方程组等号右边的数排成的向量。

    \textbf{赶的过程}(回代过程):按照下面的式子逆序求出解$x_n\to x_{n-1}\to\cdots\to x_1$。
    \begin{align}
        & x_n = y_n \nonumber \\
        & x_i = y_i - u_i x_{i+1},\quad i = n-1, n-2, \cdots, 1
    \end{align}

    追赶法的计算量仅为$5n$次乘除法。

    \subsubsection{追赶法的代数基础(P183)}

    有如下\textbf{单位上二对角阵}
    \begin{equation}
        \mathbf{U}=
        \begin{bmatrix}
            &1  &u_1 &\cdots &\cdots &0 \\ 
            &\vdots &1 &u_2 &\cdots &\vdots \\
            &\vdots &\vdots &\ddots &\ddots &\vdots \\
            &\vdots &\cdots &\cdots &1 &u_{n-1} \\
            &0 &\cdots &\cdots &\cdots & 1
        \end{bmatrix}
    \end{equation}

    和如下\textbf{下二对角阵}
    \begin{equation}
        \mathbf{L}=
        \begin{bmatrix}
            &d_1  &\cdots &\cdots &\cdots &0 \\ 
            &a_2 &d_2 &\cdots &\cdots &\vdots \\
            &\vdots &\ddots &\ddots &\cdots &\vdots \\
            &\vdots &\cdots &a_{n-1} &d_{n-1} &\cdots \\
            &0 &\cdots &\cdots &a_n &d_n
        \end{bmatrix}
    \end{equation}
    (类似地,可以得到\textbf{单位下二对角阵}和\textbf{上二对角阵}的形式)

    \begin{theorem}
        如果矩阵(\ref{sanduijiaozhen})为对角占优,则它可以:
        \begin{itemize}
            \item 分解为$\mathbf{A}=\mathbf{LU}$的形式,其中$\mathbf{L}$为单位下二对角阵,$\mathbf{U}$为上二对角阵。此分解称为\textbf{杜里特尔(Doolittle)分解}。
            \item 分解为$\mathbf{A}=\mathbf{LU}$的形式,其中$\mathbf{L}$为下二对角阵,$\mathbf{U}$为单位上二对角阵。此分解称为\textbf{克劳特(Crout)分解}。
        \end{itemize}
        且上述两种分解都是唯一的。
    \end{theorem}

    \subsection*{平方根法}

    此为选讲内容,故略。

    \subsection{误差分析}

    \subsubsection{方程组的病态}

    系数只有微小差别,解却大不相同的方程组称作是\textbf{病态的}。

    记
    \begin{equation*}
        \text{cond}(\mathbf{A})\equiv \lVert\mathbf{A}^{-1}\rVert \lVert\mathbf{A}\rVert
    \end{equation*}
    为矩阵$\mathbf{A}$的\textbf{条件数}。则系数矩阵$\mathbf{A}$和右端项$\mathbf{b}$导致的误差估计式可以分别表示为
    \begin{equation*}
        \frac{\lVert\delta\mathbf{x}\rVert}{\lVert\mathbf{x}\rVert} \leq \text{cond}(\mathbf{A})\frac{\lVert\delta\mathbf{b}\rVert}{\lVert\mathbf{b}\rVert}
    \end{equation*}
    与
    \begin{equation*}
        \frac{\lVert\delta\mathbf{x}\rVert}{\lVert\mathbf{x}\rVert} \leq \frac{\text{cond}(\mathbf{A})\frac{\lVert\delta\mathbf{A}\rVert}{\lVert\mathbf{A}\rVert}}{1-\text{cond}(\mathbf{A})\frac{\lVert\delta\mathbf{A}\rVert}{\lVert\mathbf{A}\rVert}}
    \end{equation*}

    条件数$\text{cond}(A)$刻画了方程组“病态”的程度。

    不能用行列式值$\text{det}(A)$是否很小来衡量方程组$\mathbf{Ax}=\mathbf{b}$的病态程度。
    
    \subsubsection{精度分析}

    将近似解$\widetilde{\mathbf{x}}$代回到原方程组中得到\textbf{余量$\mathbf{r}$}:
    \begin{equation*}
        \mathbf{r}=\mathbf{b}-\mathbf{A}\widetilde{\mathbf{x}}
    \end{equation*}
    如果$\mathbf{r}$很小,就认为$\widetilde{\mathbf{x}}$是相当准确的。

\end{multicols}

\end{document}